\chapter{Grammar}

\def\pronoun#1{{\tt\colorbox{gray@v}{#1}}}
\long\def\demoline#1#2{{\parindent=0pt\relax\fbox{\hbox to\hsize{%
\hbox to.65\hsize{\hsize=.63\hsize\colorbox{gray@x}{\it\vtop{\color{google@blue}#1}}}%
\hskip.035\hsize\lower1.25ex\hbox to.3\hsize{\hsize=.29\hsize\relax\everypar{}\vtop{#2}}\hfill}}}}
\def\specific#1{{\tt\colorbox{gray@v}{\color{google@green}#1}}}
\def\outs#1{{\tt\colorbox{gray@c}{\color{google@red}#1}}}

\bgroup
% adjust paragraph
\parskip=.8ex\relax
\baselineskip=3.8ex\relax
\lineskip=.9ex\relax

\section{Clause}
\index{clause}
\makeatletter
\newbox\b@xbigstar \setbox\b@xbigstar=\hbox{\scalebox{2}{$\star$}}
\def\bigstar{{\copy\b@xbigstar}}
\makeatother

{\tt Clause is the smallest grammartic unit that can express a complete proposition.}
And clause is an important and hard part in english grammar, so today I spend some
time to learn and note something about clause.

The different types of clause include \emph{Main clause}(independent clause) and 
\emph{subodinate clause}(dependent clause).
The former can express complete meaning, and the latter are those that would be
awkward or incomplete if the were alone.

\section{subodinate clause}
\index{subodinate clause}

Firstly, we should know the kind of subodinate clause.\par
{\leftskip=3em \medskip \tt \baselineskip=.8\baselineskip
\begin{easylist}[checklist]
\ListProperties(Style*=$\Rightarrow\hskip.5em$)
# content clause (noun clause)
## declarative content clause
## interrogative content clause
# relative clause
## restrictive relative clause
## non-restrictive relative clause
## human antecedents(\textipa{\color{google@blue}[""Anti"sid@nt]})
## non-human antecedents
# adverbial clause
\end{easylist}
}

\subsection{relative clause}
\index{relative clause}

Relative clauses in the English language are formed principally by means of relative pronouns.
\index{relative pronouns}
The basic relative pronouns are {\it who, which} and {\it that}, {\it who} also has the derivedc forms
{\it whom} and {\it whose}. Various gramatical rules and style guides determine which relative
pronouns may be suitable in various situations, especially for formal settings. In some cases the relative
pronoun may be omitted and merely implied.

Relative clauses are non-essential parts of a sentene. They may add meaning, but if they are removed, 
the sentence will still funciton grammatically. There are two broad types of relative clauses in English. 
It is important to distinguish between them because it affects the choice of pronoun 
used to introduce the clause.

The basic grammatical rules for the formation of relative clauses in English are given here.
\begin{easylist}[tractatus]
\ListProperties(FinalSpace=1em, Hang=true, Style*=\color{blue})
# The basic \specific{relative pronouns} are \pronoun{who}, \pronoun{which} and \pronoun{that}.
# The \specific{relative pronoun} comes at the very start of the relative clause unless it is preceded by 
a fronted preposition. In informal use it is normal to slide the preposition to the \specific{end of the clause} 
and leave it stranded. The relative clause may start with a larger phrase containing the relative pronoun 
after a preposition.\newline
\demoline{The bed \pronoun{on which} I was lying.}{formal use, preposition precede pronoun.}
\demoline{The bed which I was lying on.}{informal use, sliding the preposition to end of clause.}
\demoline{The bed, the owner \pronoun{of which} we had seen previously, ...}{large phrase \outs{the owner of which}}
\demoline{The bed, lying on which was a small cat, ...}{large phrase \outs{lying on which}}
# \pronoun{who} is used only with its antecedent referring to a person; 
  \pronoun{which}, referring to a thing;
  \pronoun{that}, referring to either a person or thing.
# \pronoun{that} is used only in \specific{restrictive relative clauses}, and is not preceded
by a comma; but \pronoun{who} and \pronoun{which} may be used in both \specific{restrictive} and
\specific{non-restrictive clauses}, and may or may not take a comma. In some styles of formal 
English, particularly American, using \pronoun{which} in \specific{restrictive clauses} is avoided
where possible.
# \pronoun{whom} is used only when its antecedent is the \outs{object} of the 
\specific{relative clause}, but not when its antecedent is the \outs{subject} 
of a sentence or clause.
# when a \specific{preposition} in the \specific{relative clause} is paced in front,
\outs{only} \pronoun{whom} or \pronoun{which} is used, and never acceptable is \pronoun{who}
or \pronoun{that}.With informal style the \specific{preposition} is often dangled (or stranded),
not fronted, and \pronoun{who} and \pronoun{that} may also be used; and \specific{zero relative pronoun}
is frequently used in this situation.
# when \pronoun{that} is used in a restrictive relative clause that is not fronted by
a \pronoun{preposition}, and it is not the \outs{subject} of the relative clause, it may
be omitted entirely.
But any relative pronoun when used in a \specific{non-restrictive} relative clause must 
not be ommitted, nor when its preposition is fronted, nor when its antecedent is the \outs{subject} of
the \outs{relative} clause.\newline
\demoline{The dentist \pronoun{(that)} I saw ...}{\pronoun{that} is the object of relative clause, and which can be omitted.}
\demoline{My dentist, whom I saw, ...\newline My dentist, who spoke to me, ...}{In non-restrictive relative clause, pronoun must not be ommitted.}
\demoline{The dentist that saw me ...\newline The dentist who saw me ...}{pronoun is subject of relative clause}
#  The verb in relative clause takes the same \outs{persion}(firset, second, third) and \outs{number}(singular or plural) as that of the 
antecedent of the relative pronoun.\newline
\demoline{1. The people who were present ...\newline 2. I, who am normally very tolerant ...}{%
First example, the antecedent of \pronoun{who} is peope(third persion, plural), 
so ther verb to be takes its form (\outs{were}) for third person and plural number.\newline
Second example, \pronoun{who}'s antecedent is the pronoun \outs{I}(first person, singular), so the verb to be takes its form \outs{am}.}
\end{easylist}

\begin{table}[H]
\bgroup
\long\def\crr{\cr\noalign{\hrule}}
\halign\bgroup
\vrule width1pt\strut\vbox{\vfill\hbox to.12\hsize\bgroup\hfill \pronoun{\large #} \hfill\egroup\vskip4ex}\vrule &%
\bgroup\setbox0\hbox to.85\hsize\bgroup\vbox{\hsize=.8\hsize #}\egroup%
\dimen0=\ht0 \advance\dimen0 by2ex\relax \ht0=\dimen0%
\dimen0=\dp0 \advance\dimen0 by1.4ex\relax \dp0=\dimen0%
\quad\box0\egroup\vrule width1pt\cr
\noalign{\hrule depth.3pt height.7pt}
that &
\begin{enumerate}
\baselineskip=.8\baselineskip
\item used in \outs{both} human antecedent \outs{and} non-human antecedent.
\item used \outs{only} in restrictive relative clause.
\item \pronoun{that} can't be preceded by a fronted preposition.
\item \pronoun{that} can be ommitted entirely when it's not the \outs{subject} of relative clause.
\end{enumerate}
\crr
which &
\begin{enumerate}
\baselineskip=.8\baselineskip
\item only used when antecedent is \outs{non-human}.
\item can be used in non-restrictive relative clause, and mostly used in non-restrictive
\newline relative clause.
\item can be preceded by a \outs{fronted preposition}.
\end{enumerate}
\crr
who & 
\begin{enumerate}
\baselineskip=.8\baselineskip
\item used in non-restrictive relative clause
\item used only when antecedent is human
\item can't be preceded by a fronted preposition
\item can refer to \outs{subject} or \outs{object} of relative clause
\end{enumerate}
\crr
\noalign{\hrule depth.3pt height.7pt}
\egroup\egroup
\caption{Summary about relative pronoun}
\end{table}



\begin{table}[H]
\bgroup
\long\def\crr{\cr\noalign{\hrule}}
\def\true{{$\angle$}}
\def\false{{$\diamondsuit$}}
\halign\bgroup
\vrule width1pt\strut\vbox{\vskip1ex\hbox to.12\hsize\bgroup\hfill \pronoun{\large #} \hfill\egroup\vskip.5ex}\vrule &%
%
\vbox{\vskip.5ex\hbox to.11\hsize\bgroup\hfill \small\color{google@blue}#\hfill\egroup\vskip.5ex}\vrule &%
\vbox{\vskip.5ex\hbox to.13\hsize\bgroup\hfill \small\color{google@blue}#\hfill\egroup\vskip.5ex}\vrule &%
\vbox{\vskip.5ex\hbox to.11\hsize\bgroup\hfill \small\color{google@blue}#\hfill\egroup\vskip.5ex}\vrule &%
\vbox{\vskip.5ex\hbox to.11\hsize\bgroup\hfill \small\color{google@blue}#\hfill\egroup\vskip.5ex}\vrule &%
\vbox{\vskip.5ex\hbox to.11\hsize\bgroup\hfill \small\color{google@blue}#\hfill\egroup\vskip.5ex}\vrule &%
\vbox{\vskip.5ex\hbox to.11\hsize\bgroup\hfill \small\color{google@blue}#\hfill\egroup\vskip.5ex}\vrule &%
\vbox{\vskip.5ex\hbox to.11\hsize\bgroup\hfill \small\color{google@blue}#\hfill\egroup\vskip.5ex}\vrule &%
\vbox{\vskip.5ex\hbox to.11\hsize\bgroup\hfill \small\color{google@blue}#\hfill\egroup\vskip.5ex}\vrule width1pt%
%
\cr
\noalign{\hrule depth.3pt height.7pt}
Class & restrictive & non-restrictive & human & non-human 
& preposition & subject & object & ommitted\crr
%
that & \true & \false & \true & \true & \false & \true & \true & \true\crr
%
\noalign{\hrule depth.3pt height.7pt}
\egroup\egroup
\caption{Relative Pronoun Quick Summary}
\end{table}
























\section{To For Of}

Proposition "To", "For" and "Of" is frequently used in english, so we should use it correctly.

\subsection{"OF"}
\begin{easylist}[checklist]
\ListProperties(Style*=$\diamondsuit$\quad, Style1*=$\heartsuit$\quad,Style1**=\large\tt\color{google@red}, Style2**=\tt\color{google@blue})
# used for belonging to, relating to, or connected with:
## The secret of this game is that you can't ever win.
## The highlight of the show is at the end.
## The first page of the book describes the author's profile.
## Don't touch it. That's the bag of my friend's sister. $\star$
## I always dreamed of being rich and famous.
# Used to indicate reference:
## I got marry in the summer of 2000.
## This is the picture of my family.
## I got a discount of 10 percent on the purchase.
# Used to indicate amount or number:
## I drank three couple of milk.
## A large number of people gathered to protest.
## He got a perfect score of 5 in writing assignment.
\end{easylist}

\noindent Furthermore, I think I need more detailed about the definition of "OF".
\begin{easylist}[checklist]
\ListProperties( Style1*=$\heartsuit$\quad, Style1**=\Large\tt\color{google@yellow},
                 Style2*={\color{red}\tt$\diamondsuit$ Def.}\quad, Style2**=\tt\color{google@red},
                 Style3*={\color{red}\tt$\odot$ Ex.}\quad, Style3**=\tt\color{google@blue})
# preposition
## \color{red}expressing the relationship between a part and a whole
### the sleeve of his coat.
## expressing the relationship between a scale of measure and a value.
### an increase of 5 percent.
## indicating an association between two entities, typically one of belonging.
### the son of a friend.
## expressing the relationship between a direction and a point of reference.
### north of Chicago
## expressing the relationship between a general category and the thing being 
specified which belongs to such a category.
### the city of Prague.
## indicating the relationship between a verb and an indirect object.
## indicating the material or substance constituting something.
### the house was built of bricks.
## expressing time in relation to the following hour.
### it would be just a quarter of three in New York.
\end{easylist}


\subsection{"TO"}
\begin{easylist}[checklist]
\ListProperties(Style*=$\diamondsuit$\quad, Style1*=$\heartsuit$\quad,Style1**=\large\tt\color{google@red}, Style2**=\tt\color{google@blue})
# Used to indicate the place, person, or thing that someone or something moves toward, 
or the direction of something.
## I'm heading to the entrance of building.
## The package was mailed to Mr.Kim yesterday.
## All of us went to the movie theater.
## please send it to me.
# Used to indicated a limit or an ending point:
## The snow was piled up to the roof.
## The stock prices rose up to 100 dollars.
# Used to indicated relationship:
## The letter is very important to your admission.
## My answer to your question is in the envelop.
## Don't respond to every little thing in your life.
# Used to indicate a time and period:
## I word nine to six, Monday to Friday.
## It is now 10 to five. (other words, It is 4:50)
\end{easylist}

\medskip
\noindent So see the definition of "TO":
\begin{easylist}[checklist]
\ListProperties( Style1*=$\heartsuit$\quad, Style1**=\Large\tt\color{google@yellow},
                 Style2*={\color{red}\tt$\diamondsuit$ Def.}\quad, Style2**=\tt\color{google@red},
                 Style3*={\color{red}\tt$\odot$ Ex.}\quad, Style3**=\tt\color{google@blue})
# preposition
## expressin motion in the direction of (a particular location)
### walking down to the mall.
## identifying the person or thing affected.
### you were terribly unkind to her.
## identifying a particular relationship between one person and another.
### he is married to Jan's cousin.
## indicating that two things are attached.
### he had left his bike chained to a fence.
## concerning or likely to concern (something, especially something abstract)
### a threat to world peace.
## governing a phrase expressing someone's reaction to something.
### to her astonishment, he smiled.
## used to introduce the second element in a comparision
### It's nothing to what it once was.
# adverb.
## so as to be closed or nearly closed.
### he pulled the door to behind him.
##  used with the base form of a veb to indicate that the verb is in the
infinitive, in particular.
## used without a veb following when the missing verb is clearly understood.
### he asked her to come but she said she didn't want to.
# abbreviation
## turnover.
\end{easylist}


\subsection{"FOR"}

\begin{easylist}[checklist]
\ListProperties(Style*=$\diamondsuit$\quad, Style1*=$\heartsuit$\quad,Style1**=\large\tt\color{google@red}, Style2**=\tt\color{google@blue})
# Used to indicate the use of something, and purpose:
## The place is for exhibition and shows.
## I backed a cake for your birthday.
## I put a note on the door for privacy.
## She has been studying hard for the final exam.
# Used to mean because of:
## I'm so happy for you.
## We feel deeply sorry for your loss.
## For this reason, I've decided to quit the job.
# Used to indicate time or duration:
## He's been famous for many decades.
## I attended the university for one year only.
## This is all I have for today.
\end{easylist}

\medskip
\noindent Let us to chekc the definition of "FOR":
\begin{easylist}[checklist]
\ListProperties( Style1*=$\heartsuit$\quad, Style1**=\Large\tt\color{google@yellow},
                 Style2*={\color{red}\tt$\diamondsuit$ Def.}\quad, Style2**=\tt\color{google@red},
                 Style3*={\color{red}\tt$\odot$ Ex.}\quad, Style3**=\tt\color{google@blue})
# preposition
## in support of or in favor of (a person or policy).
### they voted for independence in a referendum.
\chn 他们在独立公决中投了支持票.
## affecting, with regard to, or in respect of (someone or something)
### she is responsible for the efficient running of their department.
\chn 她负责部门的高效运作.
## on behalf of or to the benefit of (someone or something) 
### these parents aren't speaking for everyone.
\chn 这些家长没有为所有人说话.
##  having (the thing mentioned) as a purpose or function, reason or cause, destination
### she is searching for enlighenment.
\chn 她正在寻找启示.
### Alieen is proud of her family for their support.
\chn 艾琳为她家人的支持感到骄傲.
## representing
### The "F" is for Fascinating.
## in place of or in exchange for (something)
### swap these two bottles for that one.
\chn 将这两瓶换成那瓶.
## in relation to the expected norm of (something)
### she was tall for her age.
## indicating the length of (a period of time or a distance)
### he was in prison for 12 years.
### he crawled for 300 yards.
## indicating an occasion in a series.
### the camcorder failed for the third time.
# conjunction
## because; since.
### he felt guity, for he knew that he bore a share of responsibility for Fanny's death.
# abbraviation
## free on rail.
## foreign
\end{easylist}



\egroup % accord with first \bgroup in this file