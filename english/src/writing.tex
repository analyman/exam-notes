\chapter{Writing}

\begin{writing}[18-8-15]{About Life}
    \noindent {\bf Dear Myself\hskip.3em:\par}
    At this year, I face many choice, which is so difficult that I can't
    quickly decide what should I do.
    Almost every choice have some unacceptable disadvantage, but I need 
    to make a choice, it's the life.
    Everyone in the world always focus some problem, and should make a choice, 
    whether he is a rich man or poor man.

    Good luck for my life. from now on, I will put myself into learning
    some discipline for a important test. I need sleep early, and balanced
    live scheduler.
\end{writing}

\bigskip

\begin{writing}[18-8-16]{Today's life}
\noindent\hbox to\hsize{\hfill{\large\bf\color{google@blue} Today is an truely hard day.}\hfill}
\smallskip
I wake up approximate at 8'clock in morning, and I was brusing and wasing my face immediately.
Then my grandma come to knock my door and ask me whether eat breakfast at home or not, 
I said I would do it myself. So approximate 9'oclock I go out for eating breakfast, at first I
decided to eat some cake, but too many people sit in the cake store so that if I did my plan, I would 
spend long time for waiting my breakfast. So I gone straight up in the small street. Finally,
I found a buns store, ({\color{google@red}My dear president is an big big Bun, LOL}), and I buy some buns, 
and it really expensive, it cost my five {\tt\color{blue} yuan}, really bad.

I brought my breakfast to home and eated it for my stomach (LOL). I taken a bottle of hot water and 
come back my room, continue to my work, but it waste my a lot of time, I don't know the ideas of \LaTeX
which really bring a lot of problem to me.

After lunch, I immediately came back to my room, continue to do this boring and hard work till now.
\end{writing}

\break

\begin{writing}[18-8-17]{Tanabata}

\noindent\hbox to\hsize{\hfill{\large\bf\color{google@blue}Tanabata.}\hfill}
\indent Tanabata(\textipa{["tEn@bAtA]}) is a chinese traditional festival in lunar calender 7-7.
It is known as chinese Valentine's Day. When this day is come at any years, a wise and beautiful 
woman who is hnown as "ZhiNv" and resident at heaven will meet a poor man who is known as "NiuLang"
at a bridge which consisted by many magpie in the legend story. But more reality version of the origin
is people admare the nature and sky.

At this day, a lot of young boy and girls will show off the love between he and his partern. I'm a single
man, so only thing I can do is walk away here and silently look the love couple. My heart always 
been hurt at this day, also in West Valentine's Day (LOL). But in my hometown, a part of family will
invite their relatives to celebrate the festival, and very lucky my famliy is the inviter at this day.

\end{writing}

\bigskip
\begin{writing}[18-8-24]{Clause}
\centerline{\Large Clause(\textipa{\color{google@blue}[clOz]})}
A very long time almost one week, I don't write anything about my life in english. 
Just because I spend my time on some unnecessary but addictive thing which inherently influent me.
So after that day, I found I can't focus on something, because I lose my desire temporarily, maybe 
permanently I lose my desire in everything.

But today I find an intresting thing -- clause, so attention please.\newline
\hbox to.8\hsize{\vbox{\hsize=.8\hsize\bigskip \pretolerance=5000 \color{google@red}
    In grammar, clause is the smallest grammar unit  that can express the a complete
    proposition(\textipa{\color{google@blue}[prAp@"zISn]}). A typical clause consist of a subject and predicate, the 
    latter typically a verb phrase, a verb with any objects and other modifiers. However, the
    subject is sometimes not said or explicit, often the case in null-subject languages if the 
    subject is retrievable from context, but it sometimes also occurs in other languages such as
    english(as in imperative sentences and non-finite clauses).

    A simple sentences usually consists of a single finite clause with a finite verb that is
    independent. More complex senetences may contain multiple clauses. Main clauses(matrix clauses, independent clauses)
    are those that can stand alone as a sentence. Subordinate clauses (embedded clauses, dependent clauses) are those 
    that would be awkward or incomplete if they were alone..
    \medskip
}}\footnote{From wikipedia. \href{https://en.wikipedia.org/wiki/Clause}{Clause}}

So clause can sort as independent clause and dependent clause or main clause and subordinate clause.
As the name suggests, independent clause can express complete meaning, it's a complete sentence. And 
the dependent clause should be a part in sentence, it can't express a complete meaning.

Let us focus on dependent clause(subordinate clause). The different types of dependent clauses include {\color{google@yellow}content clauses(noun clauses)},
{\color{google@yellow}relative(adjectival) clauses}, {\color{google@yellow} adverbial clauses}.\footnote{\href{https://en.wikipedia.org/wiki/Dependent_clause}{Dependent clause}}
\begin{easylist}[checklist]
\ListProperties(Style*=$\heartsuit\hskip.5em$)
# content clause (noun clause)\newline
    {\color{gray}
    In grammar, a content clause is a subordinate clause that provides content implied or 
    commented upon by its main clause. The term was coined by Otto Jespersen. 
    They are also known as "noun clauses". There are two main kinds of 
    content clauses: declarative content clauses (or that-clauses), which correspond to 
    declarative sentences, and interrogative content clauses, which correspond to interrogative sentences.
    }
## declarative content clauses (that-clauses)\newline
    {\color{gray}Declarative content clauses can have a number of different grammatical 
    roles. They often serve as direct objects of verbs of reporting, cognition,
    perception, and so on. In this use, the conjunction that may 
    head the clause, but is often omitted, that is, unvoiced.\footnote{
        \href{https://en.wikipedia.org/wiki/Content_clause}{Content clause} from wikipedia}}
## interrogative(\textipa{{\color{google@blue}[""Int\textschwa "rAg\textschwa tIv]}}) content clauses\newline
    {\color{gray}Interrogative content clauses, often called indirect questions, can be used in many of 
    the same ways as declarative ones; for example, they are often direct objects 
    of verbs of cognition, reporting, and perception, but here they emphasize knowledge 
    or lack of knowledge of one element of a fact.}
# relative clause (adjectival clause)\newline
    {\color{gray}
    Relative clauses in the English language are formed principally by means of relative pronouns
    . The basic relative pronouns are who, which, and that; who also 
    has the derived forms whom and whose. Various grammatical rules and style guides determine 
    which relative pronouns may be suitable in various situations, especially for formal settings. 
    In some cases the relative pronoun may be omitted and merely implied ("This is 
    the man [that] I saw", or "This is the putter he wins with").

    English also uses free relative clauses, which have no antecedent and can be formed 
    with the pronouns such as what ("I like what you've done"), and who and whoever.
    }
## restrictive relative clause
## non-restrictive relative clause\newline
    {\color{gray}
    The distinction between restrictive, or integrated, relative clauses and non-restrictive, 
    or supplementary, relative clauses in English is shown not only in speaking (through 
    prosody), but also in writing (through punctuation): a non-restrictive relative 
    clause is preceded by a pause in speech and a comma in writing, whereas 
    a restrictive clause is not.[4] Compare the following sentences, which have 
    two quite different meanings, and correspondingly two clearly distinguished intonation patterns, depending on 
    whether the commas are inserted}
# adverbial clause\newline
    {\color{gray}
    An adverbial clause is a dependent clause that functions as an adverb; that is,
    the entire clause modifies a verb, an adjective, or another adverb. 
    As with all clauses, it contains a subject and predicate, although the subject 
    as well as the (predicate) verb may sometimes be omitted and implied 

    An adverbial clause is commonly, but not always, fronted by a subordinate conjunction—sometimes 
    called a trigger word. (In the examples below the adverbial clause is italicized 
    and the subordinate conjunction is bolded.)
    }
\end{easylist}
See detail in grammar chapter.
\end{writing}