\chapter{概率空间}

%{{{ section:随机事件和随机变量
\section{随机事件和随机变量}
\newtheorem{randomized_event}[theorem_root]{Definition}

\begin{randomized_event}[随机事件]
{\bf 随机实验}的结果称为{\bf 随机事件},简称{\bf 事件}
\idxterm{sjsj@{随机事件}}
\end{randomized_event}
%}}} end section:随机事件和随机变量

%{{{ section:事件的运算
\section{事件的运算}
\theoremstyle{definition}
\newtheorem{monotoneeventcolumn}[theorem_root]{Definition}

\indent 以$A$, $B$表示事件。

\begin{enumerate}
\item{\bf 事件的包含:}\quad 事件$A$发生,则事件$B$发生。\equalwith$A\subset B$
\item{\bf 事件的和:}\quad 事件$A$或$B$发生。\equalwith$A\cup B$
\item{\bf 事件的交:}\quad 事件$A$和$B$一起发生。\equalwith$A\cap B$
\item{\bf 事件的差:}\quad 事件$A$发生$B$不发生。\equalwith$A - B$
\item{\bf 事件的逆:}\quad 事件$A$不发生。\equalwith${\bar A} = \Omega/A$
\end{enumerate}

\begin{monotoneeventcolumn}
    有事件列$(A_{n=1}^{\infty})$,$A_i\subset A_j$\equalwith$i<j$,则称$A_{n=1}^{\infty}$为单调递增; 若%
    $A_i\supset A_j$\equalwith$i<j$,则$A_{n=1}^{\infty}$为单调递减。%
    称$\lim\limits_{i\to\infty}A_i$为事件列$A_{n=1}^{\infty}$的极限。%
    两者统称单调事件列
    \idxterm{ddsjl@{单调事件列}}
\end{monotoneeventcolumn}

\newtheorem{limit_of_set}[theorem_root]{\defn}
\begin{limit_of_set}
有集合列\((A_n)_{n=1}^\infty\), 其收敛当且仅当上极限等于下级限, \(\limsup\limits_{n\to\infty}\,A_n = \liminf\limits_{n\to\infty}\, A_n\), 既有
\begin{align*}
A &= \lim\limits_{n\to\infty}\,A_n\cr
  &= \limsup\limits_{n\to\infty}\,A_n = \bigcap\limits_{n=1}^\infty\bigcup\limits_{k=n}^\infty\,A_k\cr
  &= \liminf\limits_{n\to\infty}\,A_n = \bigcup\limits_{n=1}^\infty\bigcap\limits_{k=n}^\infty\,A_k
\end{align*}
\end{limit_of_set}
%}}} end section:事件的运算

%{{{ section:基本事件空间
\section{基本事件空间}
\newtheorem{basicevent}[theorem_root]{Definition}
\newtheorem{basiceventspace}[theorem_root]{Definition}

\begin{basicevent}[基本事件]
每次实验必出现一个而且只能出现一个基本事件, 任何两个基本事件不能同时发生。
\idxterm{jbsj@{基本事件}}
\end{basicevent}

\begin{basiceventspace}[基本事件空间]
以$\omega$表示基本事件,基本事件空间$\Omega = \{\omega\}$。事件是基本事件空间的子集。只含有穷或者可数个基本事件的基本事件空间称为{\bf 离散的}。
\idxterm{jbskj@{基本事件空间}}
\end{basiceventspace}
%}}} end section:基本事件空间

%{{{ section:可测空间
\section{事件$\sigma-$代数$\cdot$可测空间}
\newtheorem{sigmaalgebra}[theorem_root]{Definition}
\newtheorem{smallestsigmaalgebraset}[theorem_root]{Definition}

\subsection{事件$\sigma-$代数}

$\Omega$是一个集合,$\mathscr{F}$是$\Omega$的一些子集的集合。

\begin{sigmaalgebra}[$\sigma$代数]
    称$\mathscr{F}$为$\Omega$中的一个$\sigma-$代数:
    \begin{enumerate}
        \item $\Omega\in\mathscr{F}$
        \item $A\in\mathscr{F}$\equalwith${\bar A}\in\mathscr{F}$
        \item 对$(A_{i=1}^{\infty})$($A_i\in\mathscr{F}$), $\bigcup\limits_{i=i}^{\infty}A_i\,\in\mathscr{F}$
    \end{enumerate}
    \idxterm{sigmads@{$\sigma-$代数}}
\end{sigmaalgebra}

\noindent$\sigma-$代数的一些性质
\begin{itemize}
    \item$\emptyset\in\mathscr{F}$
    \item 对于$(A_{i=1}^{n})$,有$\bigcup_{i=1}^nA_{i}\in\mathscr{F}$和$\bigcap_{i=1}^{n}A_i\in\mathscr{F}$
    \item 对于$(A_{i=1}^{\infty})$,有$\bigcap_{i=1}^{\infty}A_i\in\mathscr{F}$
\end{itemize}

% 单调集合类
\newtheorem{monotonic_set_class}[theorem_root]{\defn}
\begin{monotonic_set_class}
假设\msM 和 \msH 都是\mO 的一些子集的集合.
\begin{enumerate}
    \item 称\msM 为\mO 中一单调集合类, 如果对属于\msM 的任一单调序列
    $A_1, A_2,\cdots,$ 有\(\lim\limits_{n\to\infty}\,A_n\in\msM\)
    \item 称单调集合类$M(\msH)$为含\msH 的最小单调类\idxterm{zxddl@{最小单调类}}, 如果
    \(\msH\subset M(\msH)\), 而且对于任意含\msH 的单调集合类\msM, 有$M(\msH)\subset \msM$.
\end{enumerate}
\end{monotonic_set_class}

% 代数
\newtheorem{__algebra}[theorem_root]{\defn}
\begin{__algebra}
假设\msA 是\mO 的一些子集的集合, 称\msA 为\mO 中的一代数, 如果满足条件
\begin{enumerate}
\item $\mO\in\msA$
\item $A\in\msA\,\Longrightarrow\,\bar{A}\in\msA$
\item \(A_1, A_2, \cdots, A_n\in\msA\,\Longrightarrow\,\bigcup\limits_{i=1}^n\,A_i\in\msA\)
\end{enumerate}
\end{__algebra}

% relation between monotonic class of set and sigma algebra
\newtheorem{relation_monotonic_sigma_algebra}[theorem_root]{\theorem}
\begin{relation_monotonic_sigma_algebra}
在代数\msH 中, 以下两个条件等价
\begin{enumerate}
    \item \msH 是\(\sigma-\)代数
    \item \msH 是单调集类
\end{enumerate}
\end{relation_monotonic_sigma_algebra}

度量空间的完备性, 等价于单调有界序列收敛.

\subsection{Borel $\sigma-$代数}

\begin{smallestsigmaalgebraset}[给定集系最小$\sigma-$代数]
\idxterm{gdjxzxsds@{给定集系最小$\sigma-$代数}}
给定$\Omega$的集系$\mathscr{H}$的最小$\sigma-$代数$\sigma(\mathscr{H})$:
\begin{enumerate}
    \item$\mathscr{H}\in\sigma(\mathscr{H})$
    \item$\mathscr{B}$是$\Omega$上的$\sigma-$代数,且$\mathscr{H}\subset\mathscr{B}$。%
        则$\forall\,\mathscr{B}$, 有$\sigma(\mathscr{H})\subset\mathscr{B}$
\end{enumerate}
\end{smallestsigmaalgebraset}

\newtheorem{borel_sigma_algebra}[theorem_root]{\defn}
\begin{borel_sigma_algebra}
假设\(\mO = \Rs^n\), 有\(\msH = \{(x_1, x_2,\cdots,x_n):-\infty<x_i\le a_i, i = 1, 2, \cdots, m, a_i\in\Rs\}\),
则称\(\msB = \sigma(\msH)\)为$m$维 borel$\sigma-$代数.
$A\in \msB$称为 borel可测集
\end{borel_sigma_algebra}
%}}} end section:可测空间

%{{{ 概率空间
\section{概率空间}

\newtheorem{probability_measure}[theorem_root]{\defn}
\begin{probability_measure}[概率测度]
    \label{def:probability-measure}
    设\(P(A), A\in\msF\), 是定义在\mO 中$sigma-$代数\msF 上的实值函数.
    称$P(A)$为\msF 上的概率测度(简称概率), 如果它满足下列条件:\idxterm{glcd@{概率测度}}
    \begin{enumerate}
        \item 非负性:\quad \(\forall A\in\msF, 0\le p(A)\le 1\)
        \item 规范性:\quad \(p(\mO) = 1\)
        \item 完全可加性:\quad 对于任意可数个\(A_m\in\msF (m=1,2,\cdots,)\), 
        如果\(A_i\cap A_j = \emptyset, i\ne j\), 则
        \begin{align}
            P\left(\bigcup\limits_{m=1}^\infty A_m\right) = \sum\limits_{m=1}^\infty\,P(A_m)
        \end{align}
    \end{enumerate}
\end{probability_measure}

\newtheorem*{lebesgue_measurable}{\remark}
\begin{lebesgue_measurable}[lebesgue可测]
设\(E\)是\(\Rs^n\)的集合. 如果对于\(\Rs^n\)的每个子集合\(A\)都成立
\begin{align*}
m^\ast(A) = m^\ast(A\cap E) +  m^\ast(A/ E)
\end{align*}
就称\(E\)是 lebesgue可测的, 并且定义lebesgue 测度为 \(m(E)= m^\ast(E)\).\footnote{Terence Tao Analysis}
\end{lebesgue_measurable}

\subsection{概率测度的性质}
\begin{enumerate}
    \item \(P(\emptyset) = 0\)
    \begin{proof} \(\emptyset = \emptyset \cup \emptyset\), 知\(P(\emptyset) = P(\emptyset) + P(\emptyset)\), 
    故\(P(\emptyset) = 0\)\end{proof}
    \item 可加性.\quad 对于 \(A_m\in\msF(m = 1, 2, \cdots, n)\), 如果
    \(A_i\cap A_j = \emptyset, i\ne j\), 则
    \begin{align}P\left(\bigcup\limits_{m=1}^n\right) = \sum\limits_{m=1}^n\,P(A_m)\end{align}
    \item 对立事件的概率.\quad \(P(\bar{A}) = 1 - P(A)\)
    \item \(\forall A, B\in\msF, A\subset B\), then \(P(A)\le P(B)\)
    \item 半可加性.\quad 对于任意有穷个\(A_1, A_2,\cdots, A_n\in\msF\), 有
    \begin{align}P\left(\bigcup\limits_{m=1}^n\,A_m\right) \le \sum\limits_{m=1}^n\,P(A_m)\end{align}
    \item 加法公式.\idxterm{jfgs@{加法公式}}\quad \(\forall A, B\in\msF\), 有\begin{align} P(A\cup B) = P(A) + P(B) - P(A\cap B)\end{align}
    \item 一般加法公式.\quad \(\forall A_m\in\msF, m = 1, 2, \cdots, n\), 有\begin{align}P\left(\bigcup\limits_{m=1}^n\,A_m\right) = \sum\limits_{m=1}^n\,(-1)^m s_m\end{align}
    其中有
    \begin{align*}
    s_1 &= \sum\limits_{m=1}^n\, P(A_m)\cr
    s_2 &= \sum\limits_{i<j}\, P(A_iA_j)\cr
    s_3 &= \sum\limits_{i<j<k}\, P(A_iA_jA_k)\cr
    s_n &= P(A_1A_2\cdots A_n)
    \end{align*}
    \item 连续性定理.\quad 设有\(A_n\in\msF, n\in\Ns^+\), 并且\(\,\lim\limits_{n\to\infty}\,A_n = A\), 则有\(\,\lim\limits_{n\to\infty}P(A_n) = P(\lim\limits_{n\to\infty}A_n)\)
    \begin{proof}首先可以利用可数次加性证明\((A_n)_{n=1}^\infty\)是单调集合的情况(递增\(S_{i+1} = A_{i+1} / A_i\), 递减\(S_{i+1} = A_{i} / A_{i+1}\)).
        \idxterm{gllxxdl@{概率连续性定理}}
        而后设\(B_n = \bigcup\limits_{k=1}^n\bigcap\limits_{m=k}^\infty A_m\), \(B_n\)是递增集合序列, 所以有
        \begin{align}\label{equ:seqcon1}\lim\limits_{n\to\infty}P(B_n) = P(\lim\limits_{n\to\infty}B_n)\end{align}
        可以知道, \(\lim\limits_{n\to\infty}B_n = \liminf\limits_{n\to\infty}A_n = \lim\limits_{n\to\infty}A_n\),
        所以有\begin{align}\label{equ:seqcon2}P(\lim\limits_{n\to\infty}A_n) = P(\lim\limits_{n\to\infty}B_n)\end{align}
        且有\(A_n \supset B_n(A_n\supset\bigcap\limits_{m=k}^\infty A_k, k\le n)\), 根据\refto{equ:seqcon1}{Equation}和\refto{equ:seqcon2}{Equation}可以推得
        \begin{align}\label{equ:seqcon3}
            \lim\limits_{n\to\infty}P(A_n)&\ge \lim\limits_{n\to\infty}P(B_n)\cr
            &= P(\lim\limits_{n\to\infty}B_n)\cr &= P(\lim\limits_{n\to\infty}A_n)
        \end{align}
        同理可以利用\(B_n = \bigcap\limits_{k=1}^n\bigcup\limits_{m=k}^\infty A_m\)证得\(\lim\limits_{n\to\infty}P(A_n)\le P(\lim\limits_{n\to\infty}A_n)\),
        既可以得到\(\lim\limits_{n\to\infty}P(A_n) = P(\lim\limits_{n\to\infty}A_n)\).
    \end{proof}
\end{enumerate}

\newtheorem{continuous_eqaulity__}[theorem_root]{\theorem}
\begin{continuous_eqaulity__}[完全可加性的等价条件]
设有\(P:\msF\mapsto\Rs\)是\(\sigma\)代数\msF 上的任一非负实值函数, 下面两个命题等价
\begin{enumerate}
\item \(P\)具有完全可加性
\item \(p\)具有可加性和连续性
\end{enumerate}
\end{continuous_eqaulity__}

\newtheorem{probability_exploring_theorem}[theorem_root]{\theorem}
\begin{probability_exploring_theorem}[概率开拓定理]
    假设\(P_0\)是代数\msA 上的完全可加概率测度, 而\(\msF = \sigma(\msA)\)是含\msA 的最小\(\sigma-\)代数, 
    那么, 在\msF 上存在位移的概率\(P\), \(\forall\,A\in\msA\), 有\(P(A) = P_0(A)\)
\end{probability_exploring_theorem}
%}}}

%{{{ 事件的条件概率
\section{事件的条件概率}

\newtheorem{conditional_probability}[theorem_root]{\defn}
\begin{conditional_probability}[条件概率]
\idxterm{tjgl@{条件概率}}
假设\prbsp 是一概率空间, \(A, B\in\msF\)是任意二事件, 假设\(P(B) > 0\), 则,
\begin{align}
    P(A\mid B) = {P(AB)\over P(B)}
\end{align}
为在事件\(B\)出现的条件下事件\(A\)的条件概率, 或简称为事件\(A\)关于事件\(B\)的条件概率.
\end{conditional_probability}

\newtheorem{basic_properties_of__}[theorem_root]{\remark}
\begin{basic_properties_of__}[条件概率的基本性质]
对于固定的\(B\in\msF\), \(P(B)>0\), \(P(A\mid B)\)作为\(A\)的集函数, 是\msF 上的概率测度, 既有
\begin{enumerate}
    \item \(\forall\,A\in\msF\), 有\(0\le P(A\mid B) \le 1\)
    \item \(P(\Omega\mid B) = 1\)
    \item 对任意可数个\(A_m\in\msF\,(m= 1, 2, \cdots\), 若\(\,A_i\ne A_j = \emptyset\, (i\ne j)\), 则
    \begin{align}
        P\left(\bigcup\limits_{m=1}^\infty A_m \mid  B\right) = \sum\limits_{m=1}^\infty\,P(A_m\mid B)
    \end{align}
\end{enumerate}
\end{basic_properties_of__}

\newtheorem{addition_formula}[theorem_root]{\theorem}
\begin{addition_formula}[乘法公式]
    \idxterm{cfgs@{乘法公式}}
假设\(A_m\in\msF\,m = 1, 2, \cdots, n(n\ge 2)\), 是任意\(n\)个事件, 并且\(P(A_1A_2\cdots A_{n-1}>0\), 则
\begin{align}
    P(A_1A_2\cdots A_n) = &P(A_1)P(A_2\mid A_1)P(A_3\mid A_1A_2)\cdots\cr
                          &\bigcdot P(A_n\mid A_1\cdots A_{n-1})
\end{align}
\end{addition_formula}

\newtheorem{full_probability_formula}[theorem_root]{\theorem}
\begin{full_probability_formula}[全概率公式]
    \idxterm{qglgs@{全概率公式}}
    假设\(H_i\in\msF\,(i\in I)\) 为任意用穷可数个事件, 满足条件:
    \(H_i\cap H_j = \emptyset, i\ne j\); \(\bigcup\limits_{i\in I}\,H_i = \Omega\); 
    \(P(H_i) > 0\), 那么, 对于任意事件\(A\in\msF\), 有
    \begin{align}
        P(A) = \sum\limits_{i\in I}P(A\mid H_i) P(H_i)
    \end{align}
\end{full_probability_formula}

\newtheorem{bayes_formula}[theorem_root]{Lemma}
\begin{bayes_formula}[贝叶斯公式]
    假设\(H_i\in\msF, i\in I\), \(I\)可数, 满足条件:
    \(H_i \cap H_j = \emptyset, i\ne j\); \(\bigcup\limits_{i\in I} H_i = \Omega\);
    \(P(H_i) > 0\), 那么对任意事件\(A\in\msF, P(A) > 0\), 有
    \begin{align}
        P(H_k\mid A) = {P(H_k)P(A\mid H_k)\over \sum\limits_{i\in I}P(H_i)P(A\mid H_i)}
    \end{align}
\end{bayes_formula}


\section{事件的独立性}

\newtheorem{independence_of_event}[theorem_root]{\defn}
\begin{independence_of_event}[二事件的独立性]
    假设\prbsp 是一概率空间, \(A, B\in\msF\), 那么, 称二随机事件\(A\) 和\(B\) 为互相独立的, 如果
    \begin{align}
        P(AB) = P(A)P(B)
    \end{align}
\end{independence_of_event}

\newtheorem{independence_of_multiple_event}[theorem_root]{\defn}
\begin{independence_of_multiple_event}[多事件的独立性]
    假设\prbsp 是一概率空间, \(A_i\in\msF\, i\in_{I}\), 那么, 称多个随机事件\(A\) 和\(B\) 为互相独立的, 如果
    \begin{align}
        P(\bigcap\limits_{i\in I} A_i) = \prod\limits_{i\in I}P(A_i)
    \end{align}
\end{independence_of_multiple_event}

事件两两独立不可以退出全部事件独立, 对于独立的事件列, 将其中一个事件换为其对立事件, 
其仍然独立.

%}}}