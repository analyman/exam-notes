\chapter{一维随机变量}

\section{一维随机变量的定义及其分布函数} %{{{

\newtheorem{one_dimensional_random_variable}[theorem_root]{\defn}
\begin{one_dimensional_random_variable}[一维随机变量]
\idxterm{sjbl@{随机变量}}
假设\prbsp 是以概率空间, \(\xi = \xi(\omega), \omega\in\Omega\), 是定义在
\(\Omega\)上的单值实函数. 如果对于任一实数, \(\omega-\)集合\(\{\omega: \xi(\omega)\le x\}\) 是一随机
事件, 亦即
\begin{align}
    \{\omega, \xi(\omega) \le x\}\in\msF
\end{align}
就称\(\xi = \xi(\omega)\)为随机变量.
\end{one_dimensional_random_variable}

\medskip
\noindent\(P, \xi, \chi\)的函数定义:
\halign{
\strut\qquad#\hfil\quad & #\cr
    概率测度\(P\) & \(P:\msF\mapsto [0,1]\)\cr
    随机变量\(\xi\) & \(\xi:\Omega\mapsto \Rs\)\cr
    示性函数\(\chi_A\) & \(\chi:\Omega\mapsto \{0, 1\}\)\cr
}
任何事件的示性函数都是一个随机变量.

\newtheorem{indictive_function}[theorem_root]{\defn}
\begin{indictive_function}[事件的示性函数]
对于任一事件\(A\in\msF\), 有定义在\(\Omega\)上的一个函数与之相对应:
\begin{align}
    \chi_A(\omega) = \begin{cases}
        1, & \omega\in A\cr
        0, & \omega\notin A\cr
    \end{cases}
\end{align}
\end{indictive_function}

示性函数与集合存在一一对应的关系, 集合的运算及关系在示性函数中也能
得到体现.
\halign{
\strut\hskip4em\relax$#$\qquad & $\Longleftrightarrow\qquad$$#$\hfill& \quad$#$\cr
A\subset B & \chi_A \le \chi_B\cr
A\cap B & \chi_{A\cap B} = \chi_A\times\chi_B\cr
A\cup B & \chi_{A\cup B} = \chi_A + \chi_B - \chi_A\times\chi_B\cr
A\setminus B & \chi_{A\setminus B} = \chi_{A} - \chi_{A\cap B}\cr
\bar{A} & \chi_{\bar{A}} = 1 - \chi_A\cr
}

\newtheorem{about_probability_variable_and_the_event}[theorem_root]{\lemma}
\begin{about_probability_variable_and_the_event}
    假设\prbsp 是一个概率空间, \(\xi = \xi(\omega)\) 是定义在此概率空间上
    的随机变量, 而\(\msB_1\)是直线的 borel\(\sigma-\) 代数, 那么, 对于任意
    \(B\in \msB_1\), 有
    \begin{align}
    \{\omega: \xi(\omega)\in B\}\in\msF
    \end{align}
此引理说明\(\,\xi:\,\Omega\mapsto \Rs\,\)是可测函数, 即对任意定义在\Rs 上的开集\(B\in\Rs\)
有\(\xi^{-1}(B)\)是\(\Omega\)上的开集.
\end{about_probability_variable_and_the_event}

\newtheorem{distribution_function}[theorem_root]{\defn}
\begin{distribution_function}[分布函数]
\idxterm{fbhs@{分布函数}}
假设\(\xi = \xi(\omega)\) 是概率空间\prbsp 上的随机变量. 那么, 对于任意\(x\in(-\infty, +\infty), P\{\omega:\xi(\omega)\le x\}\)有意义, 
因而此概率是\(x\)的函数, 记作
\begin{align}
F_{\xi}(x) = P\{\omega:\xi(\omega)\le x\}, x\in\Rs
\end{align}
称\(F_{\xi}(x)\)为随机变量\(\xi = \xi(\omega)\)的分布函数
\end{distribution_function}

\newtheorem{some_properties_of_disfunc}[theorem_root]{\remark}
\begin{some_properties_of_disfunc}[分布函数的基本性质]
任意随机变量\(\xi = \xi(x)\)的分布函数\(F_{\xi}(x),\,x\in\Rs\), 有以下性质
\begin{enumerate}
    \item 单调不减性:\quad \(a, b\in\Rs, a<b\quad\Longrightarrow\quad F_{\xi}(a)\le F_{\xi}(b)\)
    \item 右连续性:\quad \(a,\in\Rs, \lim\limits_{x\to a^+}F_{\xi}(x) = F_{\xi}(a)\)
    \begin{proof}
        由于\(F_{\xi}(x)\)是单调不减有界函数, 所以\(\lim\limits_{x\to a^+}F_{\xi}(x)\)必收敛.
        由于\(x\in (a, +\infty)\), 故有\(x>a\), 由单调不减可得\(F_{\xi}(x)\ge F_{\xi}(a)\),
        既有\(\lim\limits_{x\to a^+}F_{\xi}(x)\ge F_{\xi}(a)\). 

        取序列\(b_n = a + {1\over n}\), 可知\(\lim\limits_{n\to\infty}b_n = a\), 且有
        \(\lim\limits_{n\to\infty}F_{\xi}(b_n) = \lim\limits_{x\to a^+}F_{\xi}(x)\).
        定义集合序列\(S_n = \{\omega:\xi(x)\le b_n\}\), \(S_n\subset S_{n+1}\), 记\(S_a = \{\omega:\xi(x)\le a\}\).
        \(S_n\)收敛, 设\(S = \lim\limits_{n\to\infty}S_n\), 由于\(S_n\supset S_a\), 所以有\(S\supseteq S_a\).
        假设\(S\ne S_a\), 即\(\exists\,\omega_s\in\Omega\), 使得\(\omega_s\in S, \omega_s\notin S_a\), 可以知道\(\xi^{-1}(\omega_s)\in (a, \infty)\,(\text{由于}\,\xi(\omega)\le x)\).
        但是由于\(\lim\limits_{n\to\infty} b_n = a\), 可知\(\exists N\in\Ns^+\), 使得\(n>N\)时, \(b_n < \xi^{-1}(\omega_s)\), 所以\(\omega_s\notin S\), 故矛盾.
        既有\(S = S_a\).

        由连续性定理, \(\lim\limits_{n\to\infty}P(S_n) = P(\lim\limits_{n\to\infty}S_n) = P(S) = P(S_a) = F_{\xi}(a)\),
        且\(\lim\limits_{x\to a^+}F_{\xi}(x) = \lim\limits_{n\to\infty}F_{\xi}(S_n) = \lim\limits_{n\to\infty}P(S_n)\), 故有
        \begin{align}
            \lim\limits_{x\to a^+}F_{\xi}(x) = F_{\xi}(a)
        \end{align}
    \end{proof}
    \item 记\(F_{\xi}(-\infty) = \lim\limits_{x\to-\infty}F_{\xi}(x),\,F_{\xi}(+\infty) = \lim\limits_{x\to+\infty}F_{\xi}(x)\), 则有
    \begin{align}
        F_{\xi}(-\infty) &= 0\cr
        F_{\xi}(+\infty) &= 1
    \end{align}
\end{enumerate}
\end{some_properties_of_disfunc}

\subsection{离散型和连续性随机变量}

\halign{
\strut\hskip3em\bf# & \quad\vtop\bgroup\noindent#\egroup\cr
离散型 & 最多取又穷个或可数个值的随机变量\(\xi:\,\mathcal{N}\mapsto\Omega,\,\left|\mathcal{N}\right|\le\left|\Ns\right|\).\cr
连续性 & 随机变量\(\xi:\,\Rs\mapsto\Omega\), \(\xi(x)\)绝对连续, 即存在非负函数\(f(x),\,x\in\Rs\), 有 \(F(x) = \int_{-\infty}^x\,f(u)\,du\)\cr
混合型 & 不是纯连续性, 也不是纯离散性.\cr
}

%}}}

\section{离散型随机变量}%{{{

\newtheorem{discrete_random_var}[theorem_root]{\theorem}
\begin{discrete_random_var}[离散型随机变量]
假设\prbsp 是一概率空间, 而\(\xi = \xi(\omega)\) 是定义在\(\Omega\)上的单值函数,
\(X = \{x_0, x_1, \cdots\}\)是\(\xi\)的一切可能值得集合. 
那么, \(\xi = \xi(\omega),\,\omega\in\Omega\), 是随机变量(\(\xi\)-可测)\footnote{changed}, 当且仅当对于
任意\(x_i\in X\), 有
\begin{align}
\label{equ:equopa}
\{\omega:\xi(\omega) = x_i\}\in\msF
\end{align}

\begin{proof}
\ \par 充分性:\quad 若\(\xi\)-可测, 那么对于任意\(x_i\in X\),都有\(\{\omega:\xi(\omega) = x_i\}\in\msF\)
即\refto{equ:equopa}{Equation}成立.

必要性:\quad 若\refto{equ:equopa}{Eqaution} 成立, 那么\(\forall S\in e^X\), 有\(\bigcup\limits_{i\in S}\{\omega: \xi(\omega) = x\}\in\msF\),
即\(\xi\)可测.
\end{proof}
\end{discrete_random_var}

\subsection{离散型随机变量的概率分布}
\newtheorem{degrade_distru}[dis_root]{\disfunc}
\newtheorem{binomial_distru}[dis_root]{\disfunc}
\newtheorem{negative_binomial_distru}[dis_root]{\disfunc}
\newtheorem{geometry_distru}[dis_root]{\disfunc}
\newtheorem{super_geometry_distru}[dis_root]{\disfunc}
\newtheorem{negative_super_geometry_distru}[dis_root]{\disfunc}
\newtheorem{poisson_distru}[dis_root]{\disfunc}

\begin{degrade_distru}[退化分布]
\idxterm{tbfb@{退化分布}}
随机变量\(\xi\)称为退化分布:
\begin{align}
P\{\xi = a\} = 1
\end{align}
分布函数为
\begin{align}
F_{\xi}(x) = \begin{cases} 0, & x<a\cr 1, & x\ge a\cr\end{cases}
\end{align}
\end{degrade_distru}

\begin{binomial_distru}[二项分布]
\idxterm{exfb@{二项分布}}
随机变量\(\xi\)称为二项分布, 参数为\((n, p),\, n\ge 1,\, 0<p<1\), \(\xi\)的值域为 \(0, 1, \cdots, n\), 并且
\begin{align}
P\{\xi = m\} = \cbnt{n}{m}pP^{m}(1-p)^{n-m},\, m = 0,1,\cdots, n.
\end{align}
以\(B(n, p)\)表示\((n,p)\)为参数的二项分布, 记\(b(m;n, p) = \cbnt{n}{m}p^{m}(1-p)^{n-m}\).
\end{binomial_distru}

\begin{negative_binomial_distru}[负二项分布]
\idxterm{fexfb@{负二项分布}}
称随机变量\(\xi\)有负二项分布, 参数\((r, p),\,r\ge 1, 0<p<1\), 值域为\(\{r, r+1, r+2, \cdots\}\)
\begin{align}
P\{\xi = m\} = \cbnt{m-1}{r-1}p^{r}(1-p)^{m-r},\,\, m = r, r+1, \cdots
\end{align}
\end{negative_binomial_distru}

\begin{geometry_distru}[几何分布]
\idxterm{jhfb@{几何分布}}
称随机变量\(\xi\)为几何分布, 参数为\(p(0<p<1)\), 值域为\(\Ns^+\)
\begin{align}
P\{\xi = m\} = p(1-p)^{m-1},\,\, m = 1, 2, \cdots
\end{align}
\end{geometry_distru}

\begin{super_geometry_distru}{超几何分布}
\idxterm{cjhfb@{超几何分布}}
称随机变量\(\xi\)为超几何分布, 参数为\((N, M, n),\,1\le M\le N,\,1\le n\le N\)
\begin{align}
P\{\xi = m\} = {\cbnt{M}{m}\cbnt{N-M}{n-m}\over \cbnt{N}{n}},\,\, m =0, 1, \cdots, \min(n, M)
\end{align}
\end{super_geometry_distru}

\begin{negative_super_geometry_distru}{负超几何分布}
\idxterm{fcjhfb@{负超几何分布}}
称随机变量\(\xi\)为负超几何分布, 参数为\((N, M, r)\)
\begin{align}
P\{\xi = m\} = {\cbnt{m-1}{r-1}\cbnt{N-m}{M-r}\over \cbnt{N}{M}},\,\, m =r, r+1, \cdots, N-M+r
\end{align}
\end{negative_super_geometry_distru}

\begin{poisson_distru}[泊松分布]
\idxterm{bsfb@{泊松分布}}
称随机变量\(\xi\)有泊松分布, 参数为\(\lambda > 0\)
\begin{align}
P\{\xi = m\} = {\lambda^m\over m!}e^{-\lambda},\,\, m = 0, 1, 2, \cdots
\end{align}
对于泊松分布, 有
\begin{align*}
\sum\limits_{m = 0}^\infty\,{\lambda^m\over m!}e^{-\lambda} = e^\lambda e^{-\lambda} = 1
\end{align*}
\end{poisson_distru}

\newtheorem{poisson_theorem}[theorem_root]{\theorem}
\begin{poisson_theorem}[泊松定理]
记\(b(m; n,p_n) = \cbnt{n}{m}p^m_n(1-p_n)^{n-m}\), 其中参数满足条件
\(\lim\limits_{n\to\infty}np_n = \lambda > 0\). 那么, 对于任意正整数\(m\ge 1\),
有
\begin{align}
\lim\limits_{n\to\infty}b(m; n, p_n) = {\lambda^m\over m!}e^{-\lambda}
\end{align}
\begin{proof}
由\(\lim\limits_{n\to\infty}np_n = \lambda > 0\). 因而
\begin{align}
    np_n &= \lambda + o(1)\cr
    p_n = {\lambda\over n} + o\left({1\over n}\right)
\end{align}
于是对于任意\(m\ge 0\), 有
\begin{align}
    b(m; n, p) &= \cbnt{n}{m}pP^m_n(1-p_n)^P{n-m}\cr
    &= {n(n-1)\cdots(n-m+1)\over m!}\left[{\lambda\over n} + o\left(1\over n\right)\right]^m\bigcdot\left[1-{\lambda\over n} + o\left({1\over n}\right)\right]^{n-m}\cr
    &= {n(n-1)\cdots(n-m+1)\over m!}{\left[\lambda + o\left(1\right)\right]^m\over n^m}\bigcdot{\left[1-{\lambda\over n} + o\left({1\over n}\right)\right]^{n}\over\left[1-{\lambda\over n} + o\left({1\over n}\right)\right]^{m}}\cr
    &= {\left[\lambda + o(1)\right]^m\over m!}\left[1-{\lambda\over n} + o\left({1\over n}\right)\right]^n\bigcdot{1\cdot\left(1-{1\over n}\right)\cdots\left(1-{m - 1\over n}\right)\over \left[1-{\lambda\over n} + o\left({1\over n}\right)\right]^m}
\end{align}
有以下极限
\begin{align}
&\lim\limits_{n\to\infty}{\left[\lambda + o(1)\right]^m\over m!} = {\lambda^m\over m!}\cr
&\lim\limits_{n\to\infty}\left[1 - {\lambda\over n} + o(1)\right]^n = e^{-\lambda}\cr
&\lim\limits_{n\to\infty}{1\cdot\left(1 - {1\over n}\right)\cdots\left(1 - {m - 1\over n}\right)\over \left[1 - {\lambda\over n} + o\left({1\over n}\right)\right]^m} = 1
\end{align}
由此, 可以得到
\begin{align}
\lim\limits_{n\to\infty}b(m;n,p) = {\lambda^m\over m!}e^{-\lambda}
\end{align}
\end{proof}
\end{poisson_theorem}

\newtheorem*{poisson_event_flow}{}
\begin{poisson_event_flow}[泊松事件流]
\idxterm{bssjl@{泊松事件流}}
考虑事件在某个时间区间出现的次数, 以\(\nu(a, b]\)表示事件在时间区间\((a , b]\)出现的次数.
并且记\(\nu(t) = v(0, t]\).
称事件流为泊松时间流, 如果满足以下条件
\begin{enumerate}
\item 平稳性:\quad 事件\(A\)在任意事件区间\((a,a+t]\)上出现\(m, (m\ge 0)\)次的概率
\begin{align}
    p_m(t) &= P\{\nu(a, a+t) = m\}\cr
    &= P\{\nu(0, t) = m\}
\end{align}
\item 稀有性:\quad 在\(\Delta t\)事件内, 事件\(A\)出现两次及以上的概率
为\(\Delta t\)的高阶无穷小
\begin{align}
P\{\nu(\Delta t) > 1\} = o(\Delta t)
\end{align}
\item 无后效性:\quad 设有\(a_i, b_i\in\Rs^+,\, a_i < b_i\,\, i = 1, 2, \cdots n\), 那么事件\(\{\nu(a_i, b_i] = m_j\}\)相互独立.
\item 非平凡性:\quad 排除两种极端情况
\begin{align}
    &P\{\nu(t) = 0\} \not\equiv 0\cr
    &P\{\nu(t) = 0\} \not\equiv 1
\end{align}
\end{enumerate}
\end{poisson_event_flow}

\newtheorem{about_poisson_event}[theorem_root]{\theorem}
\begin{about_poisson_event}
对于泊松事件流, 在长为\(t\)的时间内某事件\(A\)出现的次数\(\nu(t)\)服从
参数为\(\lambda t\)的泊松分布, 其中\(\lambda > 0\)是常数.
即存在常数\(\lambda > 0\), 使
\begin{align}
    P\{\nu(t) = m\} = {(\lambda t)^m\over m!}e^{-\lambda t}\qquad m = 0, 1, 2, \cdots
\end{align}

\begin{proof}
    长...
\end{proof}
\end{about_poisson_event}
%}}}

\section{连续性随机变量}%{{{

\newtheorem{cont_distr_var}[theorem_root]{\defn}
\begin{cont_distr_var}
假设\(\xi = \xi(\omega)\)是定义在概率空间\prbsp 上的随机变量, 称
\(\xi\)为连续性的, 如果它的分布函数\(F_\xi(x) = P\{\xi\le x\},\,\, x\in\Rs\), 
绝对连续, 即存在一非负可积函数\(f_\xi(x),\,\,x\in\Rs\), 有
\begin{align}
F_\xi(x) = \int\limits_{-\infty}^{x}\,f_\xi(u)\,du
\end{align}
\(f_\xi(x)\)称为概率密度函数.
\end{cont_distr_var}

\newtheorem{uniform_distru}[dis_root]{\disfunc}
\newtheorem{exponential_distru}[dis_root]{\disfunc}
\newtheorem{gauss_distru}[dis_root]{\disfunc}
\newtheorem{logarithm_distru}[dis_root]{\disfunc}
\newtheorem{gamma_distru}[dis_root]{\disfunc}
\newtheorem{beta_distru}[dis_root]{\disfunc}

\begin{uniform_distru}[均匀分布]
\idxterm{jyfb@{均匀分布}}
称随机变量\(\xi\)在区间\([1, b]\)上有均匀分布, 如果它有密度函数
\begin{align}
f_\xi(x) = \begin{cases}{1\over b - 1}, & x\in [a,b]\cr 0, & x\notin[a, b]\end{cases}
\end{align}
\end{uniform_distru}

\begin{exponential_distru}[指数分布]
\idxterm{zsfb@{指数分布}}
称随机变量\(\xi\)有指数分布, 参数为\(\lambda > 0\)如果它有密度函数
\begin{align}
f_\xi(x) = \begin{cases}0, & x<0\cr \lambda e^{-\lambda x}, & x\ge 0\end{cases}\end{align}
显然有\(f_\xi(x)\ge 0,\,\int\limits_{-\infty}^{\infty}\,f_\xi(x)\,dx = 1\), 即
\begin{align*}
\int\limits_{-\infty}^{\infty}\,f_\xi(x)\,dx &= \int\limits_{0}^{\infty}\lambda e^{-\lambda x}\,dx\cr
&= (-\lambda{1\over\lambda}e^{-\lambda x})\mid_{0}^{\infty} = 1
\end{align*}
分布函数为
\begin{align}
F_\xi(x) = \begin{cases}0, & x<0\cr 1 - e^{-\lambda x}, & x\ge 0\end{cases}
\end{align}
考虑\(\int\limits_{\Rs}\,xf_\xi(x)\,dx\), 有
\begin{align}
\int\limits_{-\infty}^{\infty}\,xf_\xi(x)\,dx &= \int\limits_{-\infty}^{\infty}\,x\cdot \lambda e^{-\lambda x}\,dx
= \int\limits_{0}^{\infty}\,x\cdot \lambda e^{-\lambda x}\,dx\cr
&= {1\over \lambda}\int\limits_{0}^{\infty}\,(- \lambda x)\cdot e^{-\lambda x}\,d(-\lambda x)
= {1\over \lambda}\int\limits_{0}^{-\infty}\,x\cdot e^x\,dx\cr
&= {1\over \lambda}\int\limits_{0}^{-\infty}\,x\,d(e^x)
= {1\over \lambda}\left[\left(xe^x\right)\mid_{0}^{-\infty} - \int\limits_{0}^{-\infty}\,e^x\,dx\right]\cr
&= {1\over \lambda}\left[0 - (e^x)\mid_{0}^{-\infty}\right]
= {1\over \lambda}\left[0 - (0 - 1)\right]
= {1\over\lambda}
\end{align}
\end{exponential_distru}

\begin{gauss_distru}[正态分布]
称随机变量\(\xi\)有正态分布, 参数为\((\mu, \sigma^2),\,\,\sigma>0\)如果它有密度函数
\begin{align}
\varphi(x; \mu, \sigma^2) = {1\over\sqrt{2\pi}\sigma}e^{-{(x-\mu)^2\over 2\sigma^2}},\qquad x\in\Rs
\end{align}
正态分布函数为
\begin{align}
\Phi(x; \mu, \sigma^2) = {1\over\sqrt{2\pi}\sigma}\int\limits_{-\infty}^{x}\,e^{-{(u-\mu)^2\over 2\sigma^2}}\,du
\end{align}
考虑\(\int\limits_{\Rs}\,xf_\xi(x)\,dx\), 有
\begin{align}
N = \int\limits_{-\infty}^{\infty}\,\varphi(x; \mu, \sigma^2)\,dx &= 
   {1\over\sqrt{2\pi}\sigma}\int\limits_{-\infty}^{\infty}\,e^{-{(x-\mu)^2\over 2\sigma^2}}\,dx\cr
&= {1\over\sqrt{2\pi}\sigma}\cdot{\sqrt{2}\sigma}\int\limits_{-\infty}^{\infty}\,e^{-{(x-\mu)^2\over 2\sigma^2}}\,d({x-\mu\over\sqrt{2}\sigma})\cr
&= {1\over\sqrt{\pi}}\int\limits_{-\infty}^{\infty}\,e^{-x^2}\,dx
\end{align}
考虑\(N^2 = \left[{1\over\sqrt{\pi}}\int\limits_{-\infty}^{\infty}\,e^{-x^2}\,dx\right]^2\),
\begin{align}
\left[{1\over\sqrt{\pi}}\int\limits_{-\infty}^{\infty}\,e^{-x^2}\,dx\right]^2 &= 
\left[{1\over\sqrt{\pi}}\int\limits_{-\infty}^{\infty}\,e^{-x^2}\,dx\right]\cdot
\left[{1\over\sqrt{\pi}}\int\limits_{-\infty}^{\infty}\,e^{-y^2}\,dy\right]\cr
&= {1\over\pi}\int\limits_{-\infty}^{\infty}\int\limits_{-\infty}^{\infty}\,e^{-(x^2 + y^2)}\,dx\;dy = 
   {1\over\pi}\iint\limits_{\Rs^2}\,e^{-(x^2 + y^2)}\,ds\cr
&= {1\over\pi}\iint\limits_{\Rs^2}\,e^{-r^2}\,ds = 
   {1\over\pi}\int\limits_{0}^{2\pi}\int\limits_{0}^{\infty}\,r\cdot e^{-r^2}\,dr\;d\theta\cr
&= {1\over\pi}\int\limits_{0}^{2\pi}\int\limits_{0}^{\infty}\, e^{-r^2}\,(-{1\over 2})d(-r^2)\;d\theta =
    {1\over 2\pi}\int\limits_{0}^{2\pi}\,(-e^{-r^2})\mid_{0}^{\infty}d\theta =
    {1\over 2\pi}\int\limits_{0}^{2\pi}\,(0 - (-1))d\theta\cr
&= {1\over 2\pi}\cdot 2\pi = 1
\end{align}
且\(N > 1\), 故\(N = 1\)
\end{gauss_distru}

标准正态概率密度函数为\(\varphi(x) = {1\over 2\sqrt{\pi}}e^{-{x^2\over 2}}\), 即\(\mu = 0, \sigma = 1\).
标准正态概率密度函数是对称函数. 任意参数\(\mu, \sigma^2\)的正态概率密度\(\varphi(x; \mu, \sigma^2)\)和
分布函数\(\Phi(x; \mu, \sigma^2)\)都可以用标准正态分布的概率密度和分布函数表示.

\newtheorem{about_normal_distru}[theorem_root]{\lemma}
\begin{about_normal_distru}
如果\(\xi \sim N(\mu, \sigma^2)\), 则\(\xi^\ast = {\xi -\mu\over\sigma}\sim N(0,1)\), 并有
\begin{align}
&\Phi(x; \mu, \sigma^2) = \Phi\left({x - \mu\over\sigma}\right)\cr
&\varphi(x; \mu, \sigma^2) = {1\over\sigma}\varphi\left({x-\mu\over\sigma}\right)
\end{align}
\end{about_normal_distru}

\begin{logarithm_distru}[对数正态分布]
\idxterm{duztfb@{对数正态分布}}
称随机变量\(\xi\)有对数正态分布, 参数为\((\mu, \sigma^2),\,\,\sigma>0\)如果它有密度函数
\begin{align}
f_{\xi}(x) = \begin{cases}{1\over x\sqrt{2\pi}\sigma}e^{-{(\ln x - \mu)^2\over 2\sigma^2}}, & x>0\cr
0, & x\le 0\end{cases}
\end{align}
有\(f_{\xi}(x)\ge 0,\,\,\int\limits_{-\infty}^{\infty}\,f_{\xi}(x)\,dx = 1\). 如果随机变量\(\eta = \ln\xi \sim N(\mu, \sigma^2)\), 那么\(\xi\)有对数正态分布.
\begin{align*}
    P\{{\eta}\le x\} &= \Phi(x; \mu, \sigma^2) = P\{\ln\xi\le x\}\cr
     &= P\{\xi\le e^x\}\cr
    \Downarrow\cr
    P\{\xi\le y\} &= \Phi(\ln y; \mu, \sigma^2)
\end{align*}
\end{logarithm_distru}

\begin{gamma_distru}[Gamma Distribution]
\idxterm{gammafb@{Gamma分布}}
称随机变量\(\xi\)有\(\Gamma-\)分布, 参数为\((\alpha, \lambda),\,\,\alpha>0,\lambda>0\)如果它有密度函数
\begin{align}
f_{\xi}(x) = \begin{cases}{\lambda^\alpha\over\Gamma(\alpha)}x^{\alpha - 1}e^{-\lambda x}, & x>0\cr
0, & x\le 0\end{cases}
\end{align}
显然有\(f_\xi(x)\ge 0\), 并且\(\int\limits_{-\infty}^{\infty}\,f_\xi(x)\,dx = 1\). 用\(\Gamma(\alpha, \lambda)\)表示
参数为\(\alpha, \lambda)\)的\(\Gamma-\)分布, \(\xi\sim\Gamma(\alpha, \lambda)\)表示随机变量服从\(\Gamma-\)分布.
\end{gamma_distru}

\idxterm{zsfb@{指数分布}}\idxterm{elfb@{Erlang分布}}\idxterm{chifb@{\(\chi^2\)分布}}
指数分布, Erlang分布, 和\(\chi^2\)分布, 可以通过\((\alpha,\lambda)\)的特殊取值得到.

\begin{beta_distru}[Beta Distribution]
\idxterm{betafb@{Beta分布}}
称随机变量\(\xi\)有\(B-\)分布, 参数为\((\alpha, \beta),\,\,\alpha>0,\beta>0\)如果它有密度函数
\begin{align}
f_{\xi}(x) = \begin{cases}{1\over B(\alpha, \beta)}x^{\alpha - 1}(1-x)^{\beta - 1}, & x\in(0,1)\cr
0, & x\notin(0, 1)\end{cases}
\end{align}
其中
\begin{align}
    B(\alpha, \beta) = \int\limits_{0}^{1}\,x^{\alpha - 1}(1-x)^{\beta - 1}\,dx
\end{align}
\end{beta_distru}

\newtheorem{snedecor_distribution}[dis_root]{\disfunc}
\begin{snedecor_distribution}[\(F-\)分布]
\idxterm{ffb@{\(F-\)分布}}
称随机变量\(\eta\)有\(F-\)分布, 参数为\((m, n)\)如果它有密度函数
\begin{align}
f_{\xi}(x) = \begin{cases}{m^{m\over 2}n^{n\over 2}\over B({m\over 2}, {n\over 2})}x^{{m\over 2} - 1}(n+mx)^{-{m+n\over 2}}, & x>0\cr
0, & x\le 0\end{cases}
\end{align}
可以验证, 对于任意\(x>0\), \(\eta\)的分布函数\(F_{\eta}(x)\)和密度\(f_{\eta}(x)\)可以通过
\(B-\)分布函数\(F_{\xi}(x; \alpha, \beta)\)和密度\(f_{\xi}(x; \alpha, \beta)\)表示
\begin{align}
    F_{\eta}(x) &= F_{\xi}\left({mx\over n + mx}; {m\over 2}, {n\over 2}\right)\cr
    f_{\eta}(x) &= {\displaystyle mn\over(n+mx)^2}f_{\xi}\left({mx\over n + mx}; {m\over 2}, {n\over 2}\right)
\end{align}
\end{snedecor_distribution}

%}}}

\section{一维随机变量的概率分布}%{{{

\newtheorem{one_dim_random_var_distru}[theorem_root]{\defn}
\begin{one_dim_random_var_distru}[一维随机变量的概率分布]
考虑一概率空间\prbsp 和定义在此空间上的随机变量\(\xi = \xi(\omega)\).
以下两个条件等价
\begin{enumerate}
    \item 对于任意实数\(x\in\Rs\), 有
    \begin{align}\{\omega:\,\xi(\omega)\le x\}\in\msF\end{align}
    由此可知, \(\forall\,x\in\Rs\),
    \begin{align}\label{equ:equopc}F_{\xi}(x) = P\{\omega:\,\xi(\omega\le x)\}\end{align}
    有意义. 因此, \(F_{\xi}(x)\)是定义在\Rs 上的单值函数.
    \item 对于实数轴上的 borel集\(A\in\msF\), 有
    \begin{align}\{\omega:\,\xi(\omega)\in A\}\in\msF\end{align}
    由此可知, \(\forall\,A\in\msF\),
    \begin{align}\label{equ:equopb}P_{\xi}(A) = P\{\omega:\,\xi(\omega\in A)\}\end{align}
    有意义. 因此, \(P_{\xi}(A)\)是定义在\(\msB^1\) 上的单值函数, 其中\(\msB^1\)为\(\Rs^1\)的 borel \(\sigma-\)代数.
\end{enumerate}
\end{one_dim_random_var_distru}

\newtheorem{measure_of_probability__}[theorem_root]{\theorem}
\begin{measure_of_probability__}
\idxterm{glfb@{概率分布}}
由\refto{equ:equopb}{Equation}定义的集函数\(P_{\xi}(A),\,\,A\in\msB^1\), 是\(\msB^1\)上的概率测度.
\begin{proof}
证明\(P_{\xi}(x)\)是\(\msB^1\) 上的概率测度, 需要证明其满足\refto{def:probability-measure}{Definition}.
\begin{enumerate}
\item 非负性:\par
$$P\{\omega:\,\xi(\omega)\in A\}\ge 0\quad\Longrightarrow\quad P_{\xi}(A)\ge 0$$
\item 规范性:\par
$$P_{\xi}(\Rs) = P\{\omega:\,\xi(\omega)\in\Rs\} = P\{\omega:\,\omega\in\msF\} = 1$$

\item 完全可加性:\quad 对于任意\(A_m\in\msB^1,\,\,m\in\Ns\),若\(A_i\cap A_j \ne\emptyset\quad \Longleftrightarrow\quad i = j\), 则\par
\indent 设\(S_i = \{\omega:\,\xi(\omega)\in A_i\}\). 对于\(i\ne j\), 假若\(S_i\cap S_j\ne\emptyset\), 
那么有\(\omega_x\in S_i\cap S_j\), 故存在\(x = \xi(\omega_x)\in\msB^1\), 故\(x\in A_i\cap A_j\),
与前提矛盾, 故
\(S_i\cap S_j \ne\emptyset\quad \Longleftrightarrow\quad i = j\).

对于\(P_{\xi}(\bigcup\limits_{i\in\Ns}A_i)\), 有
\begin{align}
P_{\xi}(\bigcup\limits_{i\in\Ns}A_i) &= P\{\omega:\,\xi(\omega)\in\bigcup\limits_{i\in\Ns}A_i\} = P(\bigcup\limits_{i\in\Ns}S_i)\cr
&= \sum\limits_{i\in\Ns}\,P(S_i) = \sum\limits_{i\in\Ns}\,P_{\xi}(A_i)
\end{align}
即满足.
\end{enumerate}
\end{proof}
\end{measure_of_probability__}

\newtheorem{distribution_function__}[theorem_root]{\defn}
\newtheorem{probability_distribution}[theorem_root]{\defn}
\begin{distribution_function__}[分布函数]
\idxterm{fbhs@{分布函数}}
称定义在直线\(\Rs^1\)上的单值实函数\(F(x)\)为分布函数, 
如果
\begin{enumerate}
\item 单调不减, 即\(a>b\quad\Longrightarrow\quad F(a)\ge F(b)\)
\item 右连续, 即\(\lim\limits_{x\to a^+}F(x) = F(a)\)
\item \(F(-\infty) = 0,\, F(\infty) = 0\)
\end{enumerate}
\end{distribution_function__}

\begin{probability_distribution}[概率分布]
\idxterm{glfb@{概率分布}}
称定义在直线\(\Rs^1\)的 borel \(\sigma-\)代数\(\msB^1\) 上单实值集函数\(P(A)\)为概率分布,
如果它是\(\msB^1\)上的概率测度\refto{def:probability-measure}{Definition}.
\end{probability_distribution}

以上{\bf 分布函数}及{\bf 概率分布}的定义不依赖于随机变量\(\xi\).
\refto{equ:equopc}{Eqaution}和\refto{equ:equopb}{Eqaution}分别是由
随机变量\(\xi\)定义的分布函数和概率分布.

\newtheorem{relation_between_two_concept_above}[theorem_root]{\theorem}
\begin{relation_between_two_concept_above}[分布函数和概率分布一一对应]
\ \par
\begin{enumerate}
\item 对于任意概率分布\(P(A),\,A\in\msB^1\),存在唯一一个分布函数\(F(x),\,x\in\Rs\), 满足
\begin{align}
    F(x) = P((-\infty, x])
\end{align}
\item 对于任意分布函数\(F(x),\,x\in\Rs\), 存在唯一一个概率分布\(P(A),\,A\in\msB^1\), 满足
\begin{align}
    P((-\infty, x]) = F(x)
\end{align}
\end{enumerate}
\begin{proof}Too long\end{proof}
\end{relation_between_two_concept_above}

%}}}
