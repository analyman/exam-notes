\chapter{Finite Automata}

\section{Deterministic Finite Automata}

\newtheorem{deterministic_finite_state_machine}[theorem_root]{\defn}

\begin{deterministic_finite_state_machine}[finite-state machine]
A deterministic finite automation is a quintuple $M = (K,\Sigma, \delta, s, F)$ where\newline
\hbox{\quad\vtop{\begin{easylist}[tractatus]
# $K$ is a finite set of states
# $\Sigma$ is an alphabet
# $s\in K$ is the initial state
# $F\subseteq K$ is the set of final states
# $\delta$ is the transition function, $\delta: K\times\Sigma\mapsto K$
\end{easylist}}}
\end{deterministic_finite_state_machine}

Deterministic finite automata is n't allowed to move its reading head back into 
the part of the input string that has aslready been read, the portion of the 
string to the left of the reading head cann't influence the future operation of 
the machine. Thus a configureation is determined by the current state and the 
unread part of the string being processed. In other words, a {\bf configuration of a deterministic
finite automaton} $(K, \Sigma, \delta, s, F)$ is any element of $K\times\Sigma^\ast$.

The binary relation \sto holds between two configurations of $M$ if and only if
the machine can pass from one to the others as a result of a single move.
Thus if $(q, w)$ and $(q^\prime, w^\prime)$ are two configuations of $M$, then 
$(q, w)\sto (q^\prime, w^\prime)$ if and only if $w = aw^\prime$ for some symbol $a\in\Sigma$, and
$\delta(q, a) = q^\prime$. In this case we say that $(q, w)$ {\bf yields} $(q^\prime, w^\prime)$ {\bf in one step}.
Note that in fact \sto is a function $\sto: K\times\sigma^+\mapsto K\times\Sigma^\ast$.
\footnote{$\Sigma^+ = \Sigma^\ast - \emptyset$} A configuration of the form $(q, e)$ signifies 
that $M$ has consumed all its input, and hence its operation ceases at this point.

We denote the reflexive, transitive closure of \sto by $\stoc$;
$(q, w) \stoc (q^\prime, w^\prime)$ is read, $(q, w)$ yields $(q^\prime, w^\prime)$
(after some number, possibly zero, of steps).
A string $w\in\Sigma^\ast$ is said to be accepted by $M$ if and only if there is a state
$q\in F$ such that $(s, w)\stoc (q, e)$. Finally, the language accepted by $M$, $L(M)$, 
is the set of all strings accepted by $M$. 

\section{Nondeterministic Finite Automata}